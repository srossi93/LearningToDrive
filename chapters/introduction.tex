\section{Introduction}

\begin{frame}[plain]
  \begin{tikzpicture}[remember picture,overlay]
    \node[at=(current page.center)] {
      \includegraphics[width=\paperwidth]{intro_tesla.png}
    };
  \end{tikzpicture}
\end{frame}

\begin{frame}{Image Classification Pipeline}
\centering
\includegraphics[width=\textwidth]{pipeline_configuration.png}\\
\small
Classic pipeline for \textbf{Pedestrian Detection}
\end{frame}

%\begin{frame}{1. Region Proposal}
%
%Analyzes the entire frame and tries to
%\begin{itemize}
%  \item discard most of the negative regions
%  \item extract all the positive regions
%\end{itemize}
%\vspace{1mm}
%Strong influence on
%\begin{itemize}
%  \item computational efficiency
%  \item task accuracy
%\end{itemize}
%\end{frame}
%
%\begin{frame}{1. Region Proposal}
%
%\only<1>{
%\textbf{Sliding window:}
%\begin{itemize}
%  \item Simple and suitable for multiple scales
%  \item Scans the frame horizontally and vertically with a shifting window
%  \item Yields a very large number of regions
%  \item Inefficient
%\end{itemize}
%}
%
%
%\only<2>{
%\small
%\textbf{Selective search:}
%\begin{itemize}
%  \item More complex
%  \item Smaller number of candidate regions
%  \item Reduces the computational burden
%\end{itemize}
%}
%\pause
%\only<2>{
%\small
%\textbf{Locally Decorrelated Channel Features (LDCF):}
%\begin{itemize}
%  \item Ad-hoc pedestrian detection algorithm
%  \item Candidate region with a confidence value
%  \item Tradeoff between precision and recall
%\end{itemize}
%}
%
%\end{frame}
%
%
%\begin{frame}{2. Feature Extraction}
%\textbf{Input}: candidate regions\\
%\textbf{Output}: feature vector (set of real or binary values)\\
%Examples: HOG, CNN, Integral Channel Features
%\end{frame}
%
%\begin{frame}{3. Classification}
%\textbf{Input}: feature vector (set of real or binary values)\\
%\textbf{Output}: final binary response\\
%Examples: Adaboost, SVM, cross-entropy based loss functions
%\end{frame}
%
